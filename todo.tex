\documentclass[11pt]{article}
\usepackage{enumitem,amssymb,amsmath}
\usepackage{xcolor}
\usepackage{geometry}
\usepackage{graphicx}
\usepackage{hyperref}
\usepackage{pifont}
%\usepackage{fancyhdr}
\pagestyle{plain}
\markright{\hfill\today}

\newlist{todolist}{itemize}{2}
\setlist[todolist]{label=$\square$}
\newcommand{\cmark}{\ding{51}}%
\newcommand{\xmark}{\ding{55}}%
\newcommand{\done}{\rlap{$\square$}{\raisebox{2pt}{\large\hspace{1pt}\cmark}}%
\hspace{-2.5pt}}
\newcommand{\wontfix}{\rlap{$\square$}{\large\hspace{1pt}\xmark}}
%\setlength{\headheight}{15.2pt}
%\pagestyle{fancy}
%\pagestyle{headings}
%\rhead{\today}{\today}
%\markright{Sam Foreman\hfill \today}
%\pagestyle{plain}
%\fancyhf{}
%\lhead{ \plain{}{L2HMC Algorithm} }
%\rhead{ \plain{}{\today} }
%\cfoot{ \plain{}{\thepage} }


\hypersetup{
    unicode=false,          % non-Latin characters in Acrobat’s bookmarks
    colorlinks=true,        % false: boxed links; true: colored links
    linkcolor=red,          % color of internal links (change box color with linkbordercolor)
    citecolor=green,        % color of links to bibliography
    filecolor=magenta,      % color of file links
    urlcolor=cyan           % color of external links
}

\begin{document}
\noindent
So following James’ suggestion to modify the loss function to use a Gaussian
\begin{equation}
\eta_{\lambda}(\Lambda; \Lambda_{0}, \sigma_{\Lambda}) =
    \frac{\lambda}{\sqrt{2 \pi \sigma_{\Lambda}^{2}}}
        \exp\left[\frac{(\Lambda - \Lambda_{0})^{2}}{2\sigma_{\Lambda}}\right],
\end{equation}
where    
\begin{equation}
    \Lambda \equiv \Lambda(\xi, \xi^{\prime}) = \delta(\xi, \xi^{\prime}) A(\xi^{\prime}|\xi),
\end{equation}
and    
\begin{equation}
    \delta(\xi, \xi^{\prime}) = \|x - x^{\prime}\|_{2}^{2}
\end{equation}
The total loss is then computed by averaging $\eta_{\lambda}(\Lambda; \Lambda_{0}, \sigma_{\Lambda})$ over all chains.
\newline
\noindent\rule{4cm}{0.4pt}
\newline
\newline
\newline
\noindent
\textbf{\textcolor{red}{\huge TODO:}}\\
%\noindent\rule{10cm}{0.4pt}\hfill\textit{(Edited: \today)}
%\begin{itemize}
%    \item \textit{\large New items (added 09/11/2019):}
\begin{todolist}
    \item Monitor tunneling rate during training.
    \item Try increasing the batch size.
    \item Try changing the number of leapfrog steps.
    \item Try different annealing schedule.
    \item Modify the \textbf{loss function} as follows:
    \begin{enumerate}
        \item Let $\Lambda \equiv \Lambda(\xi, \xi^{\prime}) = \delta(\xi^{\prime},
            \xi) A(\xi^{\prime} | \xi)$, where $\delta(\xi^{\prime}, \xi) = \|x - x^{\prime}\|_{2}^{2}$
        \item In terms of $\Lambda$, we can write the
            $\ell_{\lambda}(\xi, \xi^{\prime}, A(\xi^{\prime}|\xi))$ (Eq. 7 from
            the L2HMC paper) as 
            \begin{equation}
                \ell_{\lambda}(\xi, \xi^{\prime}, A(\xi^{\prime}|\xi)) =
                    \frac{\lambda^{2}}{\Lambda(\xi, \xi^{\prime})} - \frac{\Lambda(\xi,
                    \xi^{\prime})}{\lambda^{2}}.
            \end{equation}
        \item We are interested to see how the model behaves if we replace
            $\ell_{\lambda}$ with a quadratic gaussian, $\eta_{\lambda}$:
            \begin{equation}
                \ell_{\lambda}{(\xi, \xi^{\prime})} 
                \,\,\longrightarrow \,\,\eta_{\lambda}(\xi, \xi^{\prime}) 
                    \equiv \exp\left[\frac{{(\Lambda -
                \Lambda_0)}^{T}{(\Lambda - \Lambda_0)}}{2\, \sigma_{\Lambda}^{2}}\right] 
            \end{equation}
            where $\Lambda_0$ is a parameter related to the ``average distance''
            between the initial and proposed configuration and
            $\sigma_{\Lambda}^{2}$ controls the width of the Gaussian.
      \end{enumerate}
\end{todolist}
%\rule{\textwidth}{0.2pt}
%\noindent\rule{\hfill}{0.4pt}\hfill%\textit{(Added 09/11/2019)}
    
    %\item \textit{\large Previous items:}
\begin{todolist}
    \item Get a reasonable estimate of the \textbf{integrated
      autocorrelation length}.
    \item Look at \href{https://arxiv.org/pdf/1206.1901.pdf}{Neal's} proof and see if there's anything in the L2HMC algorithm that might violate reversibility.
    \item Try reducing the step size $\varepsilon$ during inference by some multiplicative factor $\varepsilon \rightarrow \alpha \cdot \varepsilon$ so that the acceptance probability $p_x \rightarrow 1$.
    \begin{itemize}
        \item In doing so we need to increase the number of run steps by $1/\alpha$.
      \item See if the bias in the average plaquette scales with $\varepsilon^2$.
    \end{itemize}
    \item Try and figure out why the Jacobian isn’t (\emph{exactly}) the same for the forward and backward updates.
    \item Try removing the forward/backward masks during inference (i.e.\ run strictly forward or run strictly backward)
    \item Try eliminating either the first $x$ update (i.e. $x \rightarrow
      x^{\prime}$) or second $x$ update ($x^{\prime} \rightarrow
      x^{\prime\prime}$) when running inference by setting either
      $m^{t} = [1, 1, \ldots, 1]$ and $\bar{m}^{t} = [0, 0, \ldots, 0]$ or
      vice versa (i.e.\  removing the mask when running inference).
\end{todolist}

\vspace{1cm}
\noindent
\textbf{\textcolor{blue}{\huge Completed:}}\\
\begin{todolist}
    \item[\done] Implement reversibility checker that ensures that the reversibility condition holds (for both position and momentum)
    \item[\done] Try with `float64`. (\textbf{\textcolor{red}{Error still present.}})
    \item[\done] Try as few as 10 hidden nodes and see if the error persists. (\textbf{\textcolor{red}{It does.}})
        \item[\done] Try anti-symmetric Gaussian Mixture Model and see if the trained model is an accurate representation of the target distribution (e.g. by looking at the locations of the means) \href{https://l2hmc.slack.com/archives/CK0SMC6NS/p1567623563026900}{(link to post)}.
    \item[\done] Loop over 
        \verb+net_weights = [scale_w, translation_w, transformation_w]+
    from $[0, 0, 0]$ up to $[2, 0, 0]$, $[0, 2, 0]$, $[0, 0, 2]$, and see how the error in the average plaquette changes \href{https://l2hmc.slack.com/archives/ck0smc6ns/p1567036227006600}{(link to post)}.
\end{todolist}
\end{document}
